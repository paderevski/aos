\documentclass{article}
\usepackage{amsmath}
\usepackage[top=2cm, left=2cm, right=1cm, bottom=2cm]{geometry}
\usepackage{fancyhdr}
\usepackage{graphicx}
\usepackage{longtable}

\usepackage{amsmath,amssymb}
\usepackage{iftex}
\ifPDFTeX
  \usepackage[T1]{fontenc}
  \usepackage[utf8]{inputenc}
  \usepackage{textcomp} % provide euro and other symbols
\else % if luatex or xetex
  \usepackage{unicode-math} % this also loads fontspec
  \defaultfontfeatures{Scale=MatchLowercase}
  \defaultfontfeatures[\rmfamily]{Ligatures=TeX,Scale=1}
\fi
\usepackage{lmodern}
\ifPDFTeX\else
  % xetex/luatex font selection
\fi
% Use upquote if available, for straight quotes in verbatim environments
\IfFileExists{upquote.sty}{\usepackage{upquote}}{}
\IfFileExists{microtype.sty}{% use microtype if available
  \usepackage[]{microtype}
  \UseMicrotypeSet[protrusion]{basicmath} % disable protrusion for tt fonts
}{}
\makeatletter
\@ifundefined{KOMAClassName}{% if non-KOMA class
  \IfFileExists{parskip.sty}{%
    \usepackage{parskip}
  }{% else
    \setlength{\parindent}{0pt}
    \setlength{\parskip}{6pt plus 2pt minus 1pt}}
}{% if KOMA class
  \KOMAoptions{parskip=half}}
\makeatother
\usepackage{xcolor}
\usepackage{longtable,booktabs,array}
\usepackage{calc} % for calculating minipage widths
% Correct order of tables after \paragraph or \subparagraph
\usepackage{etoolbox}
\makeatletter
\patchcmd\longtable{\par}{\if@noskipsec\mbox{}\fi\par}{}{}
\makeatother
% Allow footnotes in longtable head/foot
\IfFileExists{footnotehyper.sty}{\usepackage{footnotehyper}}{\usepackage{footnote}}
\makesavenoteenv{longtable}
\usepackage{graphicx}
\makeatletter
\def\maxwidth{\ifdim\Gin@nat@width>\linewidth\linewidth\else\Gin@nat@width\fi}
\def\maxheight{\ifdim\Gin@nat@height>\textheight\textheight\else\Gin@nat@height\fi}
\makeatother
% Scale images if necessary, so that they will not overflow the page
% margins by default, and it is still possible to overwrite the defaults
% using explicit options in \includegraphics[width, height, ...]{}
\setkeys{Gin}{width=\maxwidth,height=\maxheight,keepaspectratio}
% Set default figure placement to htbp
\makeatletter
\def\fps@figure{htbp}
\makeatother
\setlength{\emergencystretch}{3em} % prevent overfull lines
\providecommand{\tightlist}{%
}
\setcounter{secnumdepth}{-\maxdimen} % remove section numbering
\ifLuaTeX
  \usepackage{selnolig}  % disable illegal ligatures
\fi
\usepackage{bookmark}
\usepackage{enumitem}

\IfFileExists{xurl.sty}{\usepackage{xurl}}{} % add URL line breaks if available
\urlstyle{same}
\hypersetup{
  hidelinks,
  pdfcreator={LaTeX via pandoc}}


\begin{document}
\pagestyle{fancy}
\fancyhf{}
\lfoot{Test ID: 2048}
\rfoot{Page: \thepage}
\renewcommand{\footrulewidth}{0.4pt}

\noindent
\textbf{Instructions:} For each problem, circle the letter of the best answer.
You \textbf{must show all work} for credit. Partial credit may be awarded as appropriate.

\begin{enumerate}
	\itemsep2em
	\item
	\begin{minipage}[t]{\linewidth}
		Given the function defined by \(f(x)=3 x^{5}-20 x^{3}\), find all values
of \(x\) for which the graph of \(f\) is concave up.
\vspace{1em}
		\begin{enumerate}
		\itemsep1em
			\item \(x>0\)
			\item \(-\sqrt{2}<x<0\) or \(x>\sqrt{2}\)
			\item \(-2<x<0\) or \(x>2\)
			\item \(x>\sqrt{2}\)
			\item \(-2<x<2\)
		\end{enumerate}
	\end{minipage}
	\item
	\begin{minipage}[t]{\linewidth}
		If \(f(x)=x+\dfrac{1}{x}\), then the set of values for which \(f\)
increases is
\vspace{1em}
		\begin{enumerate}
		\itemsep1em
			\item \((-\infty,-1] \cup[1, \infty)\)
			\item \([-1,1]\)
			\item \((-\infty, \infty)\)
			\item \((0, \infty)\)
			\item \((-\infty, 0) \cup(0, \infty)\)
		\end{enumerate}
	\end{minipage}
	\item
	\begin{minipage}[t]{\linewidth}
		At what values of \(x\) does \(f(x)=3 x^{5}-5 x^{3}+15\) have a relative
maximum?
\vspace{1em}
		\begin{enumerate}
		\itemsep1em
			\item -1 only
			\item o only
			\item 1 only
			\item -1 and 1 only
			\item \(-1,0\) and 1
		\end{enumerate}
	\end{minipage}
	\item
	\begin{minipage}[t]{\linewidth}
		The graph of \(y=\dfrac{-5}{x-2}\) is concave downward for all values of
\(x\) such that
\vspace{1em}
		\begin{enumerate}
		\itemsep1em
			\item \(x<0\)
			\item \(x<2\)
			\item \(x<5\)
			\item \(x>0\)
			\item \(x>2\)
		\end{enumerate}
	\end{minipage}
	\item
	\begin{minipage}[t]{\linewidth}
		If the graph of \(y=x^{3}+a x^{2}+b x-4\) has a point of inflection at
\((1,-6)\), what is the value of \(b\) ?
\vspace{1em}
		\begin{enumerate}
		\itemsep1em
			\item -3
			\item 0
			\item 1
			\item 3
		\end{enumerate}
	\end{minipage}
	\item
	\begin{minipage}[t]{\linewidth}
		The function \(f\) given by \(f(x)=x^{3}+12 x-24\) is
\vspace{1em}
		\begin{enumerate}
		\itemsep1em
			\item increasing for \(x<-2\), decreasing for \(-2<x<2\), increasing for
\(x>2\)
			\item decreasing for \({x}<0\), increasing for \({x}>0\)
			\item increasing for all \(x\)
			\item decreasing for all \(x\)
			\item decreasing for \({x}<-2\), increasing for \(-2<{x}<2\), decreasing for
\({x}>2\)
		\end{enumerate}
	\end{minipage}
	\item
	\begin{minipage}[t]{\linewidth}
		The derivative of \(f\) is \(x^{4}(x-2)(x+3)\). At how many points will
the graph of \(f\) have a relative maximum?
\vspace{1em}
		\begin{enumerate}
		\itemsep1em
			\item None
			\item One
			\item Two
			\item Three
			\item Four
		\end{enumerate}
	\end{minipage}
	\item
	\begin{minipage}[t]{\linewidth}
		If \(f(x)=x^{2} e^{x}\), then the graph of \(f\) is decreasing for all
\(x\) such that
\vspace{1em}
		\begin{enumerate}
		\itemsep1em
			\item \(x<-2\)
			\item \(-2<x<0\)
			\item \(x \succ 2\)
			\item \(x<0\)
			\item \(x>0\)
		\end{enumerate}
	\end{minipage}
	\item
	\begin{minipage}[t]{\linewidth}
		The graphs of the derivatives of the functions \(f, g\), and \(h\) are
shown below. Which of the functions \(f, g\), or \(h\) have a relative
maximum on the open interval \(a<x<b\) ?\\
\includegraphics[width=5.013333333333334in]{media/f765d5c6-c7a8-54f4-bfaa-b2393cdce2e6.jpg}
\vspace{1em}
		\begin{enumerate}
		\itemsep1em
			\item \(f\) only
			\item \(g\) only
			\item \(h\) only
			\item \(f\) and \(g\) only
			\item \(f, g\), and \(h\)
		\end{enumerate}
	\end{minipage}
	\item
	\begin{minipage}[t]{\linewidth}
		The graph of a twice-differentiable function \(f\) is shown in the
figure below.\\
\includegraphics[width=2.9833333333333334in]{media/d7b0a695-c4ff-5482-a58c-e1186654c6aa.jpg}\\
Which of the following is true?
\vspace{1em}
		\begin{enumerate}
		\itemsep1em
			\item \(f(1)<f^{\prime}(1)<f^{\prime \prime}(1)\)
			\item \(f(1)<f^{\prime \prime}(1)<f^{\prime}(1)\)
			\item \(f^{\prime}(1)<f(1)<f^{\prime \prime}(1)\)
			\item \(f^{\prime \prime}(1)<f(1)<f^{\prime}(1)\)
			\item \(f^{\prime \prime}(1)<f^{\prime}(1)<f(1)\)
		\end{enumerate}
	\end{minipage}
\end{enumerate}


\section{Free Response}

The function

$$
f(x)=\frac{1}{x^{2}-4}
$$

has first derivative

$$
f^{\prime}(x)=\frac{-2 x}{\left(x^{2}-4\right)^{2}}
$$

and second derivative

$$
f^{\prime \prime}(x)=\frac{6 x^{2}+8}{\left(x^{2}-4\right)^{3}}
$$

Sketch the graph of $f(x)$ after completing the following questions:

\begin{enumerate}
	\setlength{\itemsep}{1in}
	\item State any domain restrictions for $f(x)$
	\item Determine any critical points of $f(x)$
	\item State intervals on which $f(x)$ is increasing or decreasing
	\item State intervals on which $f(x)$ is concave up or concave down
	\item Calculate any horizontal asymptotes of $f(x)$
\end{enumerate}


\end{document}
