\documentclass{exam}
\usepackage{sagetex}

\begin{document}

\begin{center}
\fbox{\fbox{\parbox{5.5in}{\centering
Answer the questions in the spaces provided. If you run out of room
for an answer, continue on the back of the page.}}}
\end{center}

\vspace{5mm}
\makebox[0.75\textwidth]{Name and section:\enspace\hrulefill}

\vspace{5mm}
\makebox[0.75\textwidth]{Instructor’s name:\enspace\hrulefill}

\begin{sagesilent}
	x,y = var('x','y')

	def p1(f, xmin=-5, ymin=-5, xmax = 5, ymax = 5):
		  c = 10
		  xrange = list(range(xmin,xmax+1))
		  yrange = list(range(ymin,ymax+1))
		  p=implicit_plot(f,(xmin,xmax),(ymin,ymax),
			linewidth=4,fill=False,axes=True,gridlines=True,frame=False,ticks=[xrange,yrange],fontsize=12)
		  return(p)

	def hyperbola(a=1,b=1,h=0,k=0,vertical=False):
		  z = p1((x-h)^2/a^2-(y-k)^2/b^2 - 1)
		  z = z+ plot(b/a*(x-h)+k,(-5,5),color="gray",linestyle="dashed")
		  z = z+ plot(-b/a*(x-h)+k,(-5,5),color="gray",linestyle="dashed")
		  z.set_axes_range(-5,5,-5,5)
		  return z
\end{sagesilent}


\begin{questions}
\question Is it true that \(x^n + y^n = z^n\) if \(x,y,z\) and \(n\) are
positive integers?. Explain.

\question Prove that the real part of all non-trivial zeros of the function
\(\zeta(z)\) is \(\frac{1}{2}\)

\question Compute \[\int_{0}^{\infty} \frac{\sin(x)}{x}\]

\question Which of these famous physicists invented time?

\begin{choices}
 \CorrectChoice Stephen Hawking
 \choice Albert Einstein
 \choice Emmy Noether
 \choice This makes no sense
\end{choices}

\question Which of these famous physicists published a paper on Brownian Motion?

\begin{checkboxes}
 \choice Stephen Hawking
 \choice Albert Einstein
 \choice Emmy Noether
 \choice I don't know
\end{checkboxes}

\question Which is a hyperbola?

\begin{oneparchoices}
	\choice \sageplot[width=1in,height=1in]{hyperbola(1,2,1,1)}
	\choice \sageplot[width=1in,height=1in]{hyperbola(2,1,1,1)}
\	\choice \sageplot[width=1in,height=1in]{hyperbola(1,1,-2,1)}
\end{oneparchoices}

\end{questions}
\end{document}
