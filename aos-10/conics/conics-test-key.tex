\documentclass[11pt,answers]{exam}
\usepackage{sagetex}
\usepackage[top=1in,left=0.5in,right=0.5in,bottom=1in]{geometry}
\usepackage{amsmath}
\begin{document}
\pagestyle{headandfoot}
\runningheadrule
\runningheader{AOS Math 10}{Conic Sections \thepage\ of \numpages}{Feb 2, 2024}

\noindent
{\Large KEY KEY -- AOS Math 10 Conic Sections Test A }

\vspace{4ex}
\noindent
{\large February 2, 2024}




\noindent
\vspace{5mm}

\vspace{5mm}
\noindent
\makebox[0.75\textwidth]{Name and block:\enspace\hrulefill}



\newcommand{\tf}[1][{}]{%
\fillin[#1][0.5in]%
}
\newcommand{\hyperbola}[4]{%
	$\dfrac{\sage{(x-#1)^2}}{\sage{#3^2}} - \dfrac{\sage{(y-#2)^2}}{\sage{#4^2}}=1$
}

\newcommand{\ellipse}[4]{%
	$\dfrac{\sage{(x-#1)^2}}{\sage{#3^2}} + \dfrac{\sage{(y-#2)^2}}{\sage{#4^2}}=1$
}

\newcommand{\parabola}[3]{%
	$\sage{(x-#1)^2}=\sage{4*#3}(\sage{(y-#2)})$
}

\begin{sagesilent}
	x,y = var('x','y')

	def p1(f, xmin=-10, ymin=-10, xmax = 10, ymax = 10):
		  c = 10
		  xrange = list(range(xmin,xmax+1,2))
		  yrange = list(range(ymin,ymax+1,2))
		  p=implicit_plot(f,(xmin,xmax),(ymin,ymax),
			linewidth=4,fill=False,axes=True,gridlines=True,frame=False,ticks=[xrange,yrange],fontsize=12)
		  return(p)

	def hyperbola(a=1,b=1,h=0,k=0,vertical=False):
		  z = p1((x-h)^2/a^2-(y-k)^2/b^2 - 1)
		  if (vertical):
				    z = p1((y-k)^2/b^2-(x-h)^2/a^2 - 1)
		  z = z+ plot(b/a*(x-h)+k,(-10,10),color="red",linestyle="dashed")
		  z = z+ plot(-b/a*(x-h)+k,(-10,10),color="red",linestyle="dashed")
		  z.set_axes_range(-10,10,-10,10)
		  return z

	def ellipse(a=1,b=1,h=0,k=0):
		  z = p1((x-h)^2/a^2+(y-k)^2/b^2 - 1)
		  z.set_axes_range(-10,10,-10,10)
		  return z

	def parabola_v(p=1,h=0,k=0):
		  z = p1((x-h)^2 - 4*p*(y-k))
		  z = z + point((h,k+p),size=120, rgbcolor='blue')
		  z.set_axes_range(-10,10,-10,10)
		  return z

	def parabola_h(p=1,h=0,k=0):
		  z = p1((y-k)^2 - 4*p*(x-h))
		  z = z + point((h+p,k),size=120, rgbcolor='blue')
		  z.set_axes_range(-10,10,-10,10)
		  return z
\end{sagesilent}

\section*{True / False}
\begin{questions}
\question[1] \tf[T] The major axis of a (non-circular) ellipse is always longer than the minor axis.
\question[1] \tf[F] The transverse axis of a hyperbola is always longer than the conjugate axis.
\question[1] \tf[F] The foci of an ellipse are on the minor axis.
\question[1] \tf[F] The focus of the parabola $x^2=8y$ is the lowest point on the parabola.
\question[1] \tf[T] The graph of $\dfrac{x^2}{16}+\dfrac{y^2}{25}=1$ fits entirely inside the graph of $x^2+y^2=30$
\question[1] \tf[T] The directrix of a parabola is perpendicular to the axis of symmetry.
\question[1] \tf[T] The distance between two foci of an ellipse is $2c$.
\question[1] \tf[F] The eccentricity of an ellipse can be $e=1.14$.
\question[1] \tf[T] A circle is an ellipse with $a=b$.
\question[1] \tf[F] The graphs of $\dfrac{x^2}{2}-\dfrac{y^2}{3} = 1$ and $\dfrac{y^2}{2}-\dfrac{x^2}{3} = 1$ have the same asymptotes.
\end{questions}

\vspace{3ex}
\clearpage
\section*{Multiple Choice}
\noindent
\textbf{Work must be shown for credit}
\begin{questions}
	\begin{minipage}{\linewidth}
\question[3] \textbf{ Note the scale is 2 below. }Which is the graph of $\dfrac{(x-3)^2}{9}-\dfrac{(y+4)^2}{4}$?

\begin{oneparchoices}
	\choice \sageplot[width=1.5in,height=1.5in]{hyperbola(2,3,3,4)}
	\CorrectChoice \sageplot[width=1.5in,height=1.5in]{hyperbola(3,2,3,-4)}
	\choice \sageplot[width=1.5in,height=1.5in]{hyperbola(2,3,3,-4,true)}
	\\
	\choice \sageplot[width=1.5in,height=1.5in]{hyperbola(3,2,3,4)}
	\choice \sageplot[width=1.5in,height=1.5in]{hyperbola(2,3,3,-4)}
	\choice \sageplot[width=1.5in,height=1.5in]{hyperbola(3,2,-3,-4)}
\end{oneparchoices} \answerline
\end{minipage}
\question[3] \textbf{ Note the scale is 2 below. }Which is the graph of $\dfrac{(x-2)^2}{4}+\dfrac{(y+4)^2}{9}$?

\begin{oneparchoices}
	\choice \sageplot[width=1.5in,height=1.5in]{ellipse(2,3,2,4)}
	\choice \sageplot[width=1.5in,height=1.5in]{ellipse(4,9,2,-4)}
	\choice \sageplot[width=1.5in,height=1.5in]{ellipse(3,2,2,-4)}
	\\
	\choice \sageplot[width=1.5in,height=1.5in]{ellipse(2,3,-2,-4)}
	\CorrectChoice \sageplot[width=1.5in,height=1.5in]{ellipse(2,3,2,-4)}
	\choice \sageplot[width=1.5in,height=1.5in]{ellipse(3,4,2,-4)}

\end{oneparchoices} \answerline

\begin{minipage}{\linewidth}
\question[3] \textbf{ Note the scale is 2 below. }Which is the graph of $(y-4)^2 = 12(x-2)$?

\begin{oneparchoices}
	\CorrectChoice \sageplot[width=1.5in,height=1.5in]{parabola_h(3,2,4)}
	\choice \sageplot[width=1.5in,height=1.5in]{parabola_h(-6,2,4)}
	\choice \sageplot[width=1.5in,height=1.5in]{parabola_h(-6,4,2)}
	\\
	\choice \sageplot[width=1.5in,height=1.5in]{parabola_h(1,2,4)}
	\choice \sageplot[width=1.5in,height=1.5in]{parabola_h(6,2,4)}
	\choice \sageplot[width=1.5in,height=1.5in]{parabola_h(3,4,2)}
\end{oneparchoices} \vspace{1ex} \answerline
\end{minipage}
\begin{minipage}{\linewidth}
\question[3] What are the foci of the hyperbola $\dfrac{x^2}{16}-\dfrac{y^2}{12} = 1$?

\begin{choices}
	\choice $(0, \pm 2)$
	\choice $(0, \pm 2\sqrt7)$
	\CorrectChoice $(\pm 2\sqrt7,0)$
	\choice $(\pm 2,0)$
\end{choices} \answerline
\vspace{0.5in}
\end{minipage}
\begin{minipage}{\linewidth}
\question[3]  Which is \textbf{not} a vertex or co-vertex of the ellipse \ellipse{0}{2}{3}{sqrt(7)}

\begin{choices}
	\choice $(0,2-\sqrt{7})$
	\CorrectChoice $(3,\sqrt{7})$
	\choice $(-3,2)$
	\choice $(3,2)$
\end{choices} \answerline
\vspace{0.5in}

% \question[3] What is the equation of a hyperbola with vertices at $(3,-2)$ and $(-9,-2)$ and foci at $(7,-2)$ and $(-13,-2)$?

% \begin{choices}
% 	\choice \hyperbola{-3}{-2}{10}{6}
% 	\choice \hyperbola{-3}{2}{8}{10}
% 	\choice \hyperbola{3}{-2}{10}{6}
% 	\choice \hyperbola{3}{-2}{6}{8}
% \end{choices} \answerline
\end{minipage}
\begin{minipage}{\linewidth}
\question[3] What is the equation of a parabola with
\begin{itemize}
	\item a vertex
 at $(3, -2)$
 \item a horizontal axis of symmetry
 \item the parabola passes through the point $(0, 1)$
\end{itemize}

 \begin{choices}
\choice \parabola{-3}{-2}{3}
\choice \parabola{3}{2}{3}
\CorrectChoice \parabola{3}{-2}{3/4}
\choice \parabola{3}{-2}{1/9}
\choice \parabola{3}{-2}{1}
\end{choices} \answerline
\vspace{0.5in}

% \question[3] Write the following conic in standard form: $4x^2 - y^2 - 24x - 4y + 16 = 0$

% \begin{choices}
% 	\choice \hyperbola{3}{-2}{2}{4}
% 	\choice \hyperbola{3}{-2}{4}{2}
% 	\choice \hyperbola{3}{2}{2}{4}
% 	\choice \hyperbola{-3}{2}{2}{4}
% \end{choices} \answerline
\end{minipage}
\begin{minipage}{\linewidth}
\question[3] Write the equation of the ellipse that has a major axis 28 units long and is parallel to the $y$ axis,
 a minor axis 26 units long, and a center at $(11, 8)$.
 \vspace{1ex}

\begin{choices}
	\choice \ellipse{-11}{-8}{14}{13}
	\choice \ellipse{11}{8}{14}{13}
	\choice \ellipse{-11}{-8}{13}{14}
	\CorrectChoice \ellipse{11}{8}{13}{14}
\end{choices} \answerline
\vspace{0.5in}

\end{minipage}
\begin{minipage}{\linewidth}
\question[3] Given the equation of a circle in standard form: $(x + 3)^2 + (y - 4)^2 = 49$. Write the equation in general form.
\vspace{1ex}
\begin{choices}
	\choice $x^2+y^2-24=0$
	\CorrectChoice $x^2+y^2+6x-8y-24=0$
	\choice $x^2+y^2-+3x-4y-24=0$
	\choice $x^2+y^2+74 = 0$
	\choice $x^2+y^2+6x-8y+74=0$
\end{choices} \answerline
\vspace{0.5in}

\end{minipage}
\end{questions}

\end{document}
