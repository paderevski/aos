\documentclass{exam}
\usepackage{sagetex}
\usepackage[margin=0.5in]{geometry}
\usepackage{amsmath}
\begin{document}
\pagestyle{headandfoot}
\runningheadrule
\runningheader{AOS Math 10}
{Conic Sections \thepage\ of \numpages}
{Feb 2, 2024}

\noindent
{\Large AOS Math 10 Conic Sections Test}
\vspace{2ex}

\noindent
\vspace{5mm}
\makebox[0.75\textwidth]{Name and section:\enspace\hrulefill}

\noindent
\vspace{5mm}
\makebox[0.75\textwidth]{Instructor’s name:\enspace\hrulefill}


\newcommand{\tf}[1][{}]{%
\fillin[#1][0.25in]%
}
\newcommand{\hyperbola}[4]{%
	$\dfrac{\sage{(x-#1)^2}}{\sage{#3^2}} - \dfrac{\sage{(y-#2)^2}}{\sage{#4^2}}$
}

\newcommand{\ellipse}[4]{%
	$\dfrac{\sage{(x-#1)^2}}{\sage{#3^2}} + \dfrac{\sage{(y-#2)^2}}{\sage{#4^2}}$
}
\begin{sagesilent}
	x,y = var('x','y')

	def p1(f, xmin=-10, ymin=-10, xmax = 10, ymax = 10):
		  c = 10
		  xrange = list(range(xmin,xmax+1,2))
		  yrange = list(range(ymin,ymax+1,2))
		  p=implicit_plot(f,(xmin,xmax),(ymin,ymax),
			linewidth=4,fill=False,axes=True,gridlines=True,frame=False,ticks=[xrange,yrange],fontsize=12)
		  return(p)

	def hyperbola(a=1,b=1,h=0,k=0,vertical=False):
		  z = p1((x-h)^2/a^2-(y-k)^2/b^2 - 1)
		  z = z+ plot(b/a*(x-h)+k,(-5,5),color="gray",linestyle="dashed")
		  z = z+ plot(-b/a*(x-h)+k,(-5,5),color="gray",linestyle="dashed")
		  z.set_axes_range(-10,10,-10,10)
		  return z

	def ellipse(a=1,b=1,h=0,k=0):
		  z = p1((x-h)^2/a^2+(y-k)^2/b^2 - 1)
		  z.set_axes_range(-10,10,-10,10)
		  return z

	def parabola_v(p=1,h=0,k=0):
		  z = p1((x-h)^2 - 4*p*(y-k))
		  z = z + point((h,k+p),size=60, rgbcolor='blue')
		  z.set_axes_range(-10,10,-10,10)
		  return z
	def parabola_h(p=1,h=0,k=0):
		  z = p1((y-k)^2 - 4*p*(x-h))
		  z = z + point((h+p,k),size=60, rgbcolor='blue')
		  z.set_axes_range(-10,10,-10,10)
		  return z
\end{sagesilent}


\section*{True / False}
\begin{questions}
\question \tf[T] The major axis of a (non-circular) ellipse is always longer than the minor axis.
\question \tf[F] The transverse axis of a hyperbola is always longer than the conjugate axis.
\question \tf[F] The foci of an ellipse are on the minor axis.
\question \tf[F] The focus of the parabola $x^2=8y$ is the lowest point on the parabola.
\question \tf[T] The graph of $\dfrac{x^2}{16}+\dfrac{y^2}{25}=1$ fits entirely inside the graph of $x^2+y^2=30$
\question \tf[T] The directrix of a parabola is perpendicular to the axis of symmetry.
\question \tf[T] The distance between two foci of an ellipse or a hyperbola is $2c$.
\question \tf[T] The eccentricity of an ellipse can be $e=1.14$.
\question \tf[T] A circle is just an ellipse with $a=b$.
\question \tf[F] The graphs of $\dfrac{x^2}{2}-\dfrac{y^2}{3} = 1$ and $\dfrac{y^2}{2}-\dfrac{x^2}{3} = 1$ have the same asymptotes.
\end{questions}

\vspace{3ex}
\section*{Multiple Choice}
\begin{questions}
\question Which is the graph of $\dfrac{(x-3)^2}{9}-\dfrac{(y+4)^2}{4}$?

\begin{oneparchoices}
	\choice \sageplot[width=1.5in,height=1.5in]{hyperbola(3,2,3,-4)}
	\choice \sageplot[width=1.5in,height=1.5in]{hyperbola(3,2,-3,-4)}
	\choice \sageplot[width=1.5in,height=1.5in]{hyperbola(3,2,3,4)}
	\\
	\choice \sageplot[width=1.5in,height=1.5in]{hyperbola(2,3,3,4)}
	\choice \sageplot[width=1.5in,height=1.5in]{hyperbola(2,3,3,4)}
	\choice \sageplot[width=1.5in,height=1.5in]{hyperbola(2,3,0,0)}
\end{oneparchoices} \answerline

\question Which is the graph of $\dfrac{(x-2)^2}{4}+\dfrac{(y+4)^2}{9}$?

\begin{oneparchoices}
	\choice \sageplot[width=1.5in,height=1.5in]{ellipse(2,3,2,-4)}
	\choice \sageplot[width=1.5in,height=1.5in]{ellipse(2,3,-2,-4)}
	\choice \sageplot[width=1.5in,height=1.5in]{ellipse(2,3,2,4)}
	\\
	\choice \sageplot[width=1.5in,height=1.5in]{ellipse(3,4,2,-4)}
	\choice \sageplot[width=1.5in,height=1.5in]{ellipse(3,2,2,-4)}
	\choice \sageplot[width=1.5in,height=1.5in]{ellipse(4,9,2,-4)}

\end{oneparchoices} \answerline

\begin{minipage}{\textwidth}
\question Which is the graph of $(y-4)^2 = 12(x-2)$?

\begin{oneparchoices}
	\choice \sageplot[width=1.5in,height=1.5in]{parabola_h(3,2,4)}
	\choice \sageplot[width=1.5in,height=1.5in]{parabola_h(-6,2,4)}
	\choice \sageplot[width=1.5in,height=1.5in]{parabola_h(3,4,2)}
	\\
	\choice \sageplot[width=1.5in,height=1.5in]{parabola_h(6,2,4)}
	\choice \sageplot[width=1.5in,height=1.5in]{parabola_h(1,2,4)}
	\choice \sageplot[width=1.5in,height=1.5in]{parabola_h(-6,4,2)}

\end{oneparchoices}
\answerline
\end{minipage}

\question What are the foci of the hyperbola $\dfrac{x^2}{16}-\dfrac{y^2}{12} = 1$?

\begin{oneparchoices}
	\choice $(\pm 2\sqrt7,0)$
	\choice $(\pm 2,0)$
	\choice $(0, \pm 2)$
	\choice $(0, \pm 2\sqrt7)$
\end{oneparchoices} \answerline

\question  What are the vertices of the ellipse \ellipse{0}{2}{3}{sqrt(7)}

\begin{oneparchoices}
	\choice $(\pm 3,0)$
	\choice $(0,\pm3)$
	\choice $(2 \pm \sqrt7, 0)$
	\choice $(0, 2 \pm \sqrt7)$
\end{oneparchoices} \answerline

\question What is the equation of a hyperbola with vertices at $(3,-2)$ and $(-9,-2)$ and foci at $(7,-2)$ and $(-13,-2)$?

\begin{oneparchoices}
	\choice $x$
	\choice $x$
	\choice $x$
	\choice $x$
\end{oneparchoices} \answerline

\question What is the equation of a parabola where the vertex is $(3, -2)$ that passes through the point
 $(0, 1)$ and has a horizontal axis of symmetry?

 \begin{oneparchoices}
	\choice $x$
	\choice $x$
	\choice $x$
	\choice $x$
\end{oneparchoices} \answerline

\question Write the following conic in standard form: $4x^2 - y^2 - 24x - 4y + 16 = 0$

\begin{oneparchoices}
	\choice $x$
	\choice $x$
	\choice $x$
	\choice $x$
\end{oneparchoices} \answerline

\question Write the equation of the ellipse that has a major axis 28 units long and is parallel to the $y$ axis,
 a minor axis 26 units long, and a center at $(11, 8)$.

\begin{oneparchoices}
	\choice $x$
	\choice $x$
	\choice $x$
	\choice $x$
\end{oneparchoices} \answerline

\question Given the equation of a circle in standard form: $(x + 3)^2 + (y - 4)^2 = 49$. Write the equation in general form.

\begin{oneparchoices}
	\choice $x$
	\choice $x$
	\choice $x$
	\choice $x$
\end{oneparchoices} \answerline


\end{questions}

\end{document}
