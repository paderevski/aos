\documentclass{exam}
\usepackage{sagetex}
\usepackage[margin=0.5in]{geometry}
\usepackage{amsmath}
\begin{document}
\pagestyle{headandfoot}
\runningheadrule
\runningheader{AOS Math 10}
{Conic Sections \thepage\ of \numpages}
{Feb 2, 2024}

\noindent
{\Large AOS Math 10 Conic Sections Test}
\vspace{2ex}

\noindent
\vspace{5mm}
\makebox[0.75\textwidth]{Name and section:\enspace\hrulefill}

\noindent
\vspace{5mm}
\makebox[0.75\textwidth]{Instructor’s name:\enspace\hrulefill}

\vspace{4ex}

\newcommand{\tf}[1][{}]{%
\fillin[#1][0.25in]%
}
\newcommand{\hyperbola}[4]{%
	$\dfrac{\sage{(x-#1)^2}}{\sage{#3^2}} - \dfrac{\sage{(y-#2)^2}}{\sage{#4^2}}$
}

\begin{sagesilent}
	x,y = var('x','y')

	def p1(f, xmin=-5, ymin=-5, xmax = 5, ymax = 5):
		  c = 10
		  xrange = list(range(xmin,xmax+1))
		  yrange = list(range(ymin,ymax+1))
		  p=implicit_plot(f,(xmin,xmax),(ymin,ymax),
			linewidth=4,fill=False,axes=True,gridlines=True,frame=False,ticks=[xrange,yrange],fontsize=12)
		  return(p)

	def hyperbola(a=1,b=1,h=0,k=0,vertical=False):
		  z = p1((x-h)^2/a^2-(y-k)^2/b^2 - 1)
		  z = z+ plot(b/a*(x-h)+k,(-5,5),color="gray",linestyle="dashed")
		  z = z+ plot(-b/a*(x-h)+k,(-5,5),color="gray",linestyle="dashed")
		  z.set_axes_range(-5,5,-5,5)
		  return z
\end{sagesilent}

\section*{Multiple Choice}
\begin{questions}
\question Which is a hyperbola?

\begin{oneparchoices}
	\choice \sageplot[width=1in,height=1in]{hyperbola(1,2,1,1)}
	\choice \sageplot[width=1in,height=1in]{hyperbola(2,1,1,1)}
	\choice \sageplot[width=1in,height=1in]{hyperbola(1,1,-2,1)}
	\choice \sageplot[width=1in,height=1in]{hyperbola(1,1,-2,2)}
\end{oneparchoices} \answerline

\question What are the foci of the hyperbola $\dfrac{x^2}{16}-\dfrac{y^2}{12} = 1$?

\begin{oneparchoices}
	\choice $(\pm 2\sqrt7,0)$
	\choice $(\pm 2,0)$
	\choice $(0, \pm 2)$
	\choice $(0, \pm 2\sqrt7)$
\end{oneparchoices} \answerline

\question  Tell me about \hyperbola{0}{2}{3}{sqrt(7)}

\begin{oneparchoices}
	\choice $(\pm 2\sqrt7,0)$
	\choice $(\pm 2,0)$
	\choice $(0, \pm 2)$
	\choice $(0, \pm 2\sqrt7)$
\end{oneparchoices} \answerline

\question  Tell me about \hyperbola{0}{2}{3}{sqrt(7)}

\begin{oneparchoices}
	\choice $(\pm 2\sqrt7,0)$
	\choice $(\pm 2,0)$
	\choice $(0, \pm 2)$
	\choice $(0, \pm 2\sqrt7)$
\end{oneparchoices} \answerline

\question Tell me about \hyperbola{0}{2}{3}{sqrt(7)}

\begin{oneparchoices}
	\choice $(\pm 2\sqrt7,0)$
	\choice $(\pm 2,0)$
	\choice $(0, \pm 2)$
	\choice $(0, \pm 2\sqrt7)$
\end{oneparchoices} \answerline

\question  Tell me about \hyperbola{0}{2}{3}{sqrt(7)}

\begin{oneparchoices}
	\choice $(\pm 2\sqrt7,0)$
	\choice $(\pm 2,0)$
	\choice $(0, \pm 2)$
	\choice $(0, \pm 2\sqrt7)$
\end{oneparchoices} \answerline
\end{questions}

\section*{True / False}
\begin{questions}
\question \tf[T] The major axis of a (non-circular) ellipse is always longer than the minor axis.
\question \tf[F] The transverse axis of a hyperbola is always longer than the conjugate axis.
\question \tf[F] The foci of an ellipse are on the minor axis.
\question \tf[F] The focus of the parabola $x^2=8y$ is the lowest point on the parabola.
\question \tf[T] The graph of $\dfrac{x^2}{16}+\dfrac{y^2}{25}=1$ fits entirely inside the graph of $x^2+y^2=30$
\question \tf[T] The directrix of a parabola is perpendicular to the axis of symmetry.
\question \tf[T] The distance between two foci of an ellipse or a hyperbola is $2c$.
\question \tf[T] The eccentricity of an ellipse can be $e=1.14$.
\question \tf[T] A circle is just an ellipse with $a=b$.
\question \tf[F] The graphs of $\dfrac{x^2}{2}-\dfrac{y^2}{3} = 1$ and $\dfrac{y^2}{2}-\dfrac{x^2}{3} = 1$ have the same asymptotes.
\end{questions}

\end{document}
