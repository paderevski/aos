% Options for packages loaded elsewhere
\PassOptionsToPackage{unicode}{hyperref}
\PassOptionsToPackage{hyphens}{url}
%
\documentclass[
]{article}
\usepackage{amsmath,amssymb}
\usepackage{iftex}
\ifPDFTeX
  \usepackage[T1]{fontenc}
  \usepackage[utf8]{inputenc}
  \usepackage{textcomp} % provide euro and other symbols
\else % if luatex or xetex
  \usepackage{unicode-math} % this also loads fontspec
  \defaultfontfeatures{Scale=MatchLowercase}
  \defaultfontfeatures[\rmfamily]{Ligatures=TeX,Scale=1}
\fi
\usepackage{lmodern}
\ifPDFTeX\else
  % xetex/luatex font selection
\fi
% Use upquote if available, for straight quotes in verbatim environments
\IfFileExists{upquote.sty}{\usepackage{upquote}}{}
\IfFileExists{microtype.sty}{% use microtype if available
  \usepackage[]{microtype}
  \UseMicrotypeSet[protrusion]{basicmath} % disable protrusion for tt fonts
}{}
\makeatletter
\@ifundefined{KOMAClassName}{% if non-KOMA class
  \IfFileExists{parskip.sty}{%
    \usepackage{parskip}
  }{% else
    \setlength{\parindent}{0pt}
    \setlength{\parskip}{6pt plus 2pt minus 1pt}}
}{% if KOMA class
  \KOMAoptions{parskip=half}}
\makeatother
\usepackage{xcolor}
\usepackage{graphicx}
\makeatletter
\def\maxwidth{\ifdim\Gin@nat@width>\linewidth\linewidth\else\Gin@nat@width\fi}
\def\maxheight{\ifdim\Gin@nat@height>\textheight\textheight\else\Gin@nat@height\fi}
\makeatother
% Scale images if necessary, so that they will not overflow the page
% margins by default, and it is still possible to overwrite the defaults
% using explicit options in \includegraphics[width, height, ...]{}
\setkeys{Gin}{width=\maxwidth,height=\maxheight,keepaspectratio}
% Set default figure placement to htbp
\makeatletter
\def\fps@figure{htbp}
\makeatother
\setlength{\emergencystretch}{3em} % prevent overfull lines
\providecommand{\tightlist}{%
  \setlength{\itemsep}{0pt}\setlength{\parskip}{0pt}}
\setcounter{secnumdepth}{-\maxdimen} % remove section numbering
\ifLuaTeX
  \usepackage{selnolig}  % disable illegal ligatures
\fi
\usepackage{bookmark}
\IfFileExists{xurl.sty}{\usepackage{xurl}}{} % add URL line breaks if available
\urlstyle{same}
\hypersetup{
  hidelinks,
  pdfcreator={LaTeX via pandoc}}

\author{}
\date{}

\begin{document}

\subsection{Multiple Choice}\label{multiple-choice}

\begin{enumerate}
\def\labelenumi{\arabic{enumi}.}
\tightlist
\item
  (calculator not allowed)
  \[\lim _{n \rightarrow \infty} \frac{4 n^{2}}{n^{2}+10000 n} \text { is }\]

  \begin{enumerate}
  \def\labelenumii{(\alph{enumii})}
  \tightlist
  \item
    0
  \item
    \(\frac{1}{2500}\)
  \item
    1
  \item
    4
  \item
    nonexistent
  \end{enumerate}
\item
  (calculator not allowed) The
  \(\lim _{h \rightarrow 0} \frac{\tan 3(x+h)-\tan 3 x}{h}\) is

  \begin{enumerate}
  \def\labelenumii{(\alph{enumii})}
  \tightlist
  \item
    0
  \item
    \(3 \sec ^{2}(3 x)\)
  \item
    \(\sec ^{2}(3 x)\)
  \item
    \(3 \cot (3 x)\)
  \item
    nonexistent
  \end{enumerate}
\item
  (calculator not allowed)
  \(\lim _{x \rightarrow 0} \frac{7 x-\sin x}{x^{2}+\sin (3 x)}=\)

  \begin{enumerate}
  \def\labelenumii{(\alph{enumii})}
  \tightlist
  \item
    6
  \item
    2
  \item
    1
  \item
    0
  \end{enumerate}
\item
  (calculator not allowed) At \(x=3\), the function given by
  \(f(x)=\left\{\begin{array}{cc}x^{2}, & x<3 \\ 6 x-9, & x \geq 3\end{array}\right.\)
  is

  \begin{enumerate}
  \def\labelenumii{(\alph{enumii})}
  \tightlist
  \item
    undefined.
  \item
    continuous but not differentiable.
  \item
    differentiable but not continuous.
  \item
    neither continuous nor differentiable.
  \item
    both continuous and differentiable
  \end{enumerate}
\item
  (calculator allowed)
  \includegraphics{https://cdn.mathpix.com/cropped/2024_03_10_cab44c4bf5e69d6a0bd8g-04.jpg?height=345&width=453&top_left_y=1405&top_left_x=890}.
  The figure above shows the graph of a function \(f\) with domain
  \(0 \leq x \leq 4\). Which of the following statements are true? I.
  \(\lim _{x \rightarrow 2^{-}} f(x)\) exists II.
  \(\lim _{x \rightarrow 2^{+}} f(x)\) exists III.
  \(\lim _{x \rightarrow 2} f(x)\) exists

  \begin{enumerate}
  \def\labelenumii{(\alph{enumii})}
  \tightlist
  \item
    I only
  \item
    II only
  \item
    I and II only
  \item
    I and III only
  \item
    I, II, and III
  \end{enumerate}
\item
  (calculator not allowed) If \(f(x)=2 x^{2}+1\), then
  \(\lim _{x \rightarrow 0} \frac{f(x)-f(0)}{x^{2}}\) is

  \begin{enumerate}
  \def\labelenumii{(\alph{enumii})}
  \tightlist
  \item
    0
  \item
    1
  \item
    2
  \item
    4
  \item
    nonexistent
  \end{enumerate}
\item
  (calculator not allowed) If \(f^{\prime}(x)=\cos x\) and
  \(g^{\prime}(x)=1\) for all \(x\), and if \(f(0)=g(0)=0\), then
  \(\lim _{x \rightarrow 0} \frac{f(x)}{g(x)}\) is

  \begin{enumerate}
  \def\labelenumii{(\alph{enumii})}
  \tightlist
  \item
    \(\frac{\pi}{2}\)
  \item
    1
  \item
    0
  \item
    -1
  \item
    nonexistent
  \end{enumerate}
\item
  (calculator not allowed) \[f(x)=\left\{\begin{array}{l}
  \ln (4 x-7) \text { if } x<2 \\
  4 x-7 \text { if } x \geq 2
  \end{array}\right.\] Let \(f\) be the function defined above. Which of
  the following statements about \(f\) are true? I.
  \(\lim _{x \rightarrow 2^{-}} f(x)=\lim _{x \rightarrow 2^{+}} f(x)\)
  II.
  \(\lim _{x \rightarrow 2^{-}} f^{\prime}(x)=\lim _{x \rightarrow 2^{+}} f^{\prime}(x)\)
  III. \(f\) is differentiable at \(x=2\)

  \begin{enumerate}
  \def\labelenumii{(\alph{enumii})}
  \tightlist
  \item
    I only
  \item
    II only
  \item
    II and III only
  \item
    I, II, and III
  \end{enumerate}
\item
  (calculator not allowed) If \(\lim _{x \rightarrow a} f(x)=L\) where
  \(L\) is a real number, which of the following must be true?

  \begin{enumerate}
  \def\labelenumii{(\alph{enumii})}
  \tightlist
  \item
    \(f^{\prime}(a)\) exists.
  \item
    \(f(x)\) is continuous at \(x=a\).
  \item
    \(f(x)\) is defined at \(x=a\).
  \item
    \(f(a)=L\)
  \item
    None of the above
  \end{enumerate}
\item
  (calculator not allowed) For \(x \geq 0\), the horizontal line \(y=2\)
  is an asymptote for the graph of the function \(f\). Which of the
  following statements must be true?

  \begin{enumerate}
  \def\labelenumii{(\alph{enumii})}
  \tightlist
  \item
    \(f(0)=2\)
  \item
    \(f(x) \neq 2\) for all \(x \geq 0\)
  \item
    \(f(2)\) is undefined.
  \item
    \(\lim _{x \rightarrow 2} f(x)=\infty\)
  \item
    \(\lim _{x \rightarrow \infty} f(x)=2\)
  \end{enumerate}
\item
  (calculator not allowed) If the graph of \(y=\frac{a x+b}{x+c}\) has a
  horizontal asymptote at \(y=2\) and a vertical asymptote at \(x=-3\),
  then \(a+c=\)

  \begin{enumerate}
  \def\labelenumii{(\alph{enumii})}
  \tightlist
  \item
    -5
  \item
    -1
  \item
    0
  \item
    1
  \item
    5
  \end{enumerate}
\item
  (calculator not allowed)
  \(\lim _{x \rightarrow \infty} \frac{(2 x-1)(3-x)}{(x-1)(x+3)}\) is

  \begin{enumerate}
  \def\labelenumii{(\alph{enumii})}
  \tightlist
  \item
    -3
  \item
    -2
  \item
    2
  \item
    3
  \item
    nonexistent
  \end{enumerate}
\item
  (calculator not allowed)
  \(f(x)=\left\{\begin{array}{cc}\frac{x^{2}-4}{x-2}, & x \neq 2 \\ 1, & x=2\end{array}\right.\)
  Let \(f\) be the function defined above. Which of the following
  statements about \(f\) are true? I. \(f\) has a limit at \(x=2\).
\end{enumerate}

\begin{enumerate}
\def\labelenumi{\Roman{enumi}.}
\setcounter{enumi}{1}
\tightlist
\item
  \(f\) is continuous at \(x=2\).
\item
  \(f\) is differentiable at \(x=2\). (a) I only (a) II only (a) III
  only (a) I and II only (a) I, II, and III
  \[f(x)=\left\{\begin{array}{l}
  x+2 b \text { if } x \leq 2 \\
  a x^{2} \text { if } x>2
  \end{array}\right.\]
\end{enumerate}

\begin{enumerate}
\def\labelenumi{\arabic{enumi}.}
\setcounter{enumi}{13}
\tightlist
\item
  Let \(f\) be the function given above. What are all values of \(a\)
  and \(b\) for which \(f\) is differentiable at \(x=2\) ?

  \begin{enumerate}
  \def\labelenumii{(\alph{enumii})}
  \tightlist
  \item
    \(a=\frac{1}{4}\) and \(b=-\frac{1}{2}\)
  \item
    \(\quad a=\frac{1}{4}\) and \(b=\frac{1}{2}\)
  \item
    \(\quad a=\frac{1}{4}\) and \(b\) is any real number
  \item
    \(a=b+2\), where \(b\) is any real number
  \item
    There are no such values of \(a\) and \(b\)
  \end{enumerate}
\item
  (calculator not allowed) If the function \(f\) is continuous for all
  real numbers and if \(f(x)=\frac{x^{2}-4}{x+2}\) when \(x \neq-2\),
  then \(f(-2)=\)

  \begin{enumerate}
  \def\labelenumii{(\alph{enumii})}
  \tightlist
  \item
    -4
  \item
    -2
  \item
    -1
  \item
    0
  \item
    2
  \end{enumerate}
\item
  (calculator not allowed)
  \includegraphics{https://cdn.mathpix.com/cropped/2024_03_10_cab44c4bf5e69d6a0bd8g-08.jpg?height=507&width=675&top_left_y=359&top_left_x=709}
  The graph of a function \(f\) is shown above. At which value of
  \(\mathrm{x}\) is \(f\) continuous, but not differentiable?

  \begin{enumerate}
  \def\labelenumii{(\alph{enumii})}
  \tightlist
  \item
    \(a\)
  \item
    \(b\)
  \item
    \(c\)
  \item
    \(d\)
  \item
    \(e\)
  \end{enumerate}
\item
  (calculator allowed)
  \includegraphics{https://cdn.mathpix.com/cropped/2024_03_10_cab44c4bf5e69d6a0bd8g-08.jpg?height=352&width=1484&top_left_y=1506&top_left_x=304}
  For which of the following does \(\lim _{x \rightarrow 4} f(x)\)
  exist?

  \begin{enumerate}
  \def\labelenumii{(\alph{enumii})}
  \tightlist
  \item
    I only
  \item
    II only
  \item
    III only
  \item
    I and II only
  \item
    I and III only
  \end{enumerate}
\item
  (calculator not allowed)
  \(\lim _{x \rightarrow e} \frac{\ln x-1}{x-e}\) is

  \begin{enumerate}
  \def\labelenumii{(\alph{enumii})}
  \tightlist
  \item
    \(\frac{1}{e}\)
  \item
    1
  \item
    \(e\)
  \item
    nonexistent
    \includegraphics{https://cdn.mathpix.com/cropped/2024_03_10_cab44c4bf5e69d6a0bd8g-09.jpg?height=561&width=480&top_left_y=928&top_left_x=318}
    Graph of \(f\)
    \includegraphics{https://cdn.mathpix.com/cropped/2024_03_10_cab44c4bf5e69d6a0bd8g-09.jpg?height=563&width=482&top_left_y=930&top_left_x=884}
    Graph of \(g\)
  \end{enumerate}
\item
  The graphs of the functions \(f\) and \(g\) are shown above. The value
  of \(\lim _{x \rightarrow 1} f(g(x))\) is

  \begin{enumerate}
  \def\labelenumii{(\alph{enumii})}
  \tightlist
  \item
    1
  \item
    2
  \item
    3
  \item
    nonexistent
  \end{enumerate}
\item
  (calculator not allowed) Let \(f\) be a function defined by
  \(f(x)=\left\{\begin{array}{l}1-2 \sin x, x \leq 0 \\ e^{-4 x}, x>0\end{array}\right.\)

  \begin{enumerate}
  \def\labelenumii{(\alph{enumii})}
  \tightlist
  \item
    Show that \(f\) is continuous at \(x=0\).
  \end{enumerate}
\item
  (calculator not allowed) \(2012 \mathrm{AB} 4\) The function \(f\) is
  defined by \(f(x)=\sqrt{25-x^{2}}\) for \(-5 \leq x \leq 5\).

  \begin{enumerate}
  \def\labelenumii{(\alph{enumii})}
  \tightlist
  \item
    Let \(g\) be the function defined by
    \(g(x)=\left\{\begin{array}{l}f(x) \text { for }-5 \leq x \leq-3 \\ x+7 \text { for }-3<x<5\end{array}\right.\)
    Is \(g\) continuous at \(x=-3\) ? Use the definition of continuity
    to explain your answer.
  \end{enumerate}
\end{enumerate}

\subsection{Limits, Continuity, and Differentiability Reference
Page}\label{limits-continuity-and-differentiability-reference-page}

\subsection{Existence of a Limit at a
Point}\label{existence-of-a-limit-at-a-point}

A function \(f(x)\) has a limit \(L\) as \(x\) approaches \(C\) if and
only if the left-hand and right-hand limits at \(C\) exist and are
equal. 1. \(\lim _{x \rightarrow c^{-}} f(x)\) exists 2.
\(\lim _{x \rightarrow c^{+}} f(x)\) exists 3.
\(\lim _{x \rightarrow c^{-}} f(x)=\lim _{x \rightarrow c^{+}} f(x) \quad \therefore \lim _{x \rightarrow c} f(x)=L\)
\#\# Continuity A function is continuous on an interval if it is
continuous at every point of the interval. Intuitively, a function is
continuous if its graph can be drawn without ever needing to pick up the
pencil. This means that the graph of \(y=f(x)\) has no ``holes'', no
``jumps'' and no vertical asymptotes at \(x=a\). When answering free
response questions on the AP exam, the formal definition of continuity
is required. To earn all of the points on the free response question
scoring rubric, all three of the following criteria need to be met, with
work shown: A function is continuous at a point \(x=a\) if and only if:
1. \(f(a)\) exists 2. \(\lim _{x \rightarrow a} f(x)\) exists 3.
\(\lim _{x \rightarrow a} f(x)=f(a)\) (i.e., the limit equals the
function value) \#\# Limit Definition of a Derivative The derivative of
a function \(f(x)\) with respect to \(x\) is the function
\(f^{\prime}(x)\) whose value at \(x\) is
\(f^{\prime}(x)=\lim _{h \rightarrow 0} \frac{f(x+h)-f(x)}{h}\),
provided the limit exists. \#\# Alternative Form for Definition of a
Derivative The derivative of a function \(f(x)\) at \(x=a\) is
\(f^{\prime}(a)=\lim _{x \rightarrow a} \frac{f(x)-f(a)}{x-a}\),
provided the limit exists. \#\# Continuity and Differentiability
Differentiability implies continuity (but not necessarily vice versa) If
a function is differentiable at a point (at every point on an interval),
then it is continuous at that point (on that interval). The converse is
not always true: continuous functions may not be differentiable. It is
possible for a function to be continuous at a specific value for \(a\)
but not differentiable there. \#\# L'Hospital's Rule (returns on 2017 AB
exam) Given that \(f\) and \(g\) are differentiable functions on an open
interval \((a, b)\) containing \(c\) (except possibly at \(c\) itself),
assume that \(g^{\prime}(x) \neq 0\) for all \(x\) in the interval
(except possibly at \(c\) itself). If
\(\lim _{x \rightarrow c} \frac{f(x)}{g(x)}\) produces the indeterminate
form \(\frac{0}{0}\), then
\(\lim _{x \rightarrow c} \frac{f(x)}{g(x)}=\lim _{x \rightarrow c} \frac{f^{\prime}(x)}{g^{\prime}(x)}\)
provided the limit on the right exists (or is infinite). This result
also applies when the limit produces any one of the indeterminate forms
\(\frac{\infty}{\infty}, \frac{-\infty}{\infty}, \frac{\infty}{-\infty}\),
or \(\frac{-\infty}{-\infty}\).

\end{document}
