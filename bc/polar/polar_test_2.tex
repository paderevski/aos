\documentclass[10pt]{article}
\usepackage[utf8]{inputenc}
\usepackage[T1]{fontenc}
\usepackage{graphicx}
\usepackage[export]{adjustbox}
\graphicspath{ {./images/} }
\usepackage{amsmath}
\usepackage{amsfonts}
\usepackage{amssymb}
\usepackage[version=4]{mhchem}
\usepackage{stmaryrd}
\usepackage{bbold}

\begin{document}
\begin{center}
\includegraphics[max width=\textwidth]{2024_02_16_eb58b69572f3859be6c0g-1}
\end{center}

Honesty and integrity are the foundations of good academic work. Whether you are working on a problem set, lab report, project, presentation, or paper, do not engage in plagiarism, unauthorized collaboration, cheating, or facilitating academic dishonesty. Our expectation is for our students to be successful while being trustworthy. The honor code is not intended to be punitive, but rather a guide for all students and faculty to follow. For these reasons, the Academies of Loudoun will uphold the following Honor Code:

On my honor, I have not accepted or provided any unauthorized aid on this test, quiz, or assignment.

As an Academies of I oudoun student, you agreed to uphold the Academies Honor Code. Please write the Honor Code Pledge below and sign this document.

\section{Student Signature:}
Class: Date:

Multiple Choice. Correct work must be shown for full credit. Choose the letter for the best answer.(3 pts each)

\begin{enumerate}
  \item What is the slope of the line tangent to the polar curve $r=3 \theta$ at the point where $\theta=\frac{\pi}{2}$ ?
		\begin{enumerate}
	A) $-\frac{\pi}{2}$
B) $-\frac{2}{\pi}$
C) 0
D) 3

  \item For a certain polar curve $r=f(\theta)$, it is known that $\frac{d x}{d \theta}=\cos \cos \theta-\sin \sin \theta$ and $\frac{\mathbb{d} y}{\mathbb{d} \theta}=\sin \sin \theta+\cos \cos \theta$. What is

the value of $\frac{d^{2} y}{d x^{2}}$ at $\theta=4$ ?
		\begin{enumerate}
A) -1.420
B) 0.417
C) 1.346
D) 3.195

\begin{enumerate}
  \setcounter{enumi}{2}
  \item A particle moves in a plane so that its position at any time $\theta, 0 \leq \theta \leq 8$, is given by the polar equation $r(\theta)=5(1+\cos \theta)$. When does the particle's distance from the origin change from decreasing to increasing?
		\begin{enumerate}
	A) $\theta=0$ only
B) $\theta=\pi$ only
C) $\theta=2 \pi$ only
D) $\theta=0$ and $\theta=\pi$
E) $\theta=\pi$ and $\theta=2 \pi$

  \item The area of the region enclosed by the polar curve $r=\cos 2 \theta$ for $0 \leq \theta \leq \frac{\pi}{2}$ is
		\begin{enumerate}
	A) $\frac{\pi}{2}$
B) $\pi$
C) $\frac{\pi}{8}$
D) $\frac{\pi}{4}$
E) 1

  \item The area of one leaf of the rose $r=\sin 3 \theta$ is
		\begin{enumerate}
	A) $\frac{\pi}{12}$
B) $\frac{\pi}{6}$
C) $\frac{\pi}{4}$
D) $\frac{\pi}{3}$
E) $\frac{\pi}{2}$

  \item The area outside $r=1$ and inside $r=1+\sin \theta$ is
		\begin{enumerate}
	A) $2+\pi$
B) $2+\frac{\pi}{2}$
C) $2+\frac{\pi}{4}$
D) $2-\frac{\pi}{4}$
E) $2-\frac{\pi}{2}$

  \item The total area of the region enclosed by the polar graph of $r=\cos 3 \theta$ is
		\begin{enumerate}
	A) $\frac{\pi}{12}$
B) $\frac{\pi}{6}$
C) $\frac{\pi}{4}$
D) $\frac{\pi}{3}$
E) $\frac{\pi+\sqrt{3}}{2}$

  \item Which of the following gives the area of the region enclosed by the graph of the polar curve $r=1+\cos \theta$ ?
		\begin{enumerate}
	A) $\int_{0}^{\pi}\left(1+\cos ^{2} \theta\right) d \theta$
F) $\int_{0}^{\pi}(1+\cos \theta)^{2} d \theta$
		\begin{enumerate}G) $\int_{0}^{2 \pi}(1+\cos \thet
a) d \theta$
H) $\int_{0}^{2 \pi}(1+\cos \theta)^{2} d \theta$
l) $\frac{1}{2} \int_{0}^{2 \pi}\left(1+\cos ^{2} \theta\right) d \theta$

  \item If the function $r=f(\theta)$ is continuous and nonnegative for $0 \leq \alpha \leq \theta \leq \beta \leq 2 \pi$, then the area enclosed by the polar curve $r=f(\theta)$ and the lines $\theta=\alpha$ and $\theta=\beta$ is given by
		\begin{enumerate}
	A) $\frac{1}{2} \int_{\alpha}^{\beta} f\left(\theta^{2}\right) d \theta$
		\begin{enumerate}B) $\frac{1}{2} \int_{\alpha}^{\beta} f(\thet
a) d \theta$
C) $\frac{1}{2} \int_{\alpha}^{\beta} \theta f\left(\theta^{2}\right)$
		\begin{enumerate}D) $\frac{1}{2} \int_{\alpha}^{\beta} \theta f(\thet
a) d \theta$
E) $\frac{1}{2} \int_{\alpha}^{\beta}(f(\theta))^{2} d \theta$

  \item Which of the following integrals gives the total area of the region shared by both polar curves $r=2 \cos \theta$ and $r=2 \sin \theta$ ?

\end{enumerate}

\includegraphics[max width=\textwidth, center]{2024_02_16_eb58b69572f3859be6c0g-4}
N) $2 \int_{0}^{\frac{\pi}{4}}\left(\cos ^{2} \theta-\sin ^{2} \theta\right) d \theta$

Free Response Question. (NON CALCULATOR)

\begin{center}
\includegraphics[max width=\textwidth]{2024_02_16_eb58b69572f3859be6c0g-5}
\end{center}

The graph of the polar curve $r=1-2 \cos \theta$ for $0 \leq \theta \leq \pi$ is shown above. Let $S$ be the shaded region in the third quadrant bounded by the curve and the $x$-axis.

(		\begin{enumerate}
	a) Write an integral expression for the area of $S$.

(b) Write expressions for $\frac{d x}{d \theta}$ and $\frac{d y}{d \theta}$ in terms of $\theta$.

(c) Write an equation in terms of $x$ and $y$ for the line tangent to the graph of the polar curve at the point where $\theta=\frac{\pi}{2}$. Show the computations that lead to your answer.
(a)

(b)

(c)


\end{document}