\documentclass[11pt]{exam}
\usepackage[pdftex]{graphicx}
\usepackage{asymptote}
\usepackage[left=0.55in,right=0.55in, top=1in,bottom=1in]{geometry}
\usepackage{multicol}
\usepackage{blindtext}
\usepackage{amsmath}

\begin{document}
\firstpageheader{AOS Senior BC}{Polar Graphing Exam}{Makeup Version}
\runningheader{AOS Senior BC}{Polar Graphing Exam}{Makeup Version}
\runningheadrule
\def\asydir{asy}
\def\picsize{3inch}
\setlength\parindent{0in}
\section*{Matching -- 1 point each question}
\begin{center}
\begin{tabular}{|c|c|c|}
\hline
A \begin{asy}
	import polargrid;
	size(1.75inch);
	drawPolar(new real(real t) {return 1-3*sin(t);}, 4);
\end{asy}
&
B \begin{asy}
	import polargrid;
	size(1.75inch);
	drawPolar(new real(real t) {return 1+3*cos(t);}, 4);
\end{asy}
&
C \begin{asy}
	import polargrid;
	size(1.75inch);
	drawPolar(new real(real t) {return 3+2*sin(t);}, 5);
\end{asy}
\\ \hline
D \begin{asy}
	import polargrid;
	size(1.75inch);
	drawPolar(new real(real t) {return 3-2*sin(t);}, 5);
\end{asy}
&
E \begin{asy}
	import polargrid;
	size(1.75inch);
	drawPolar(new real(real t) {return 4+2*cos(t);}, 6);
\end{asy}
&
F \begin{asy}
	import polargrid;
	size(1.75inch);
	drawPolar(new real(real t) {return 4*sin(3*t);}, 4);
\end{asy}
\\ \hline
G \begin{asy}
	import polargrid;
	size(1.75inch);
	drawPolar(new real(real t) {return 4*cos(3*t);}, 5);
\end{asy}
&
H \begin{asy}
	import polargrid;
	size(1.75inch);
	drawPolar(new real(real t) {return 4*(sin(t)+cos(t));}, 6);
\end{asy}
&
I \begin{asy}
	import polargrid;
	size(1.75inch);
	drawPolar(new real(real t) {return 2*abs(sin(t));}, 4);
\end{asy}
\\ \hline
J \begin{asy}
	import polargrid;
	size(1.75inch);
	drawPolar(new real(real t) {return 3/cos(t);}, 5,-3.5*pi/12,3.5*pi/12);
	clip((6,6)--(-6,6)--(-6,-6)--(6,-6)--cycle);
\end{asy}
&
K \begin{asy}
	import polargrid;
	size(1.75inch);
	drawPolar(new real(real t) {return t/2;},5,0,10);
	clip((6,6)--(-6,6)--(-6,-6)--(6,-6)--cycle);
\end{asy}
&
L \begin{asy}
	import polargrid;
	size(1.75inch);
	drawPolar(new real(real t) {return 3/sin(t+0.01);}, 5, 5*pi/24,19*pi/24);
	clip((6,6)--(-6,6)--(-6,-6)--(6,-6)--cycle);
\end{asy}
\\ \hline

\end{tabular}
\end{center}

\newcommand\aline[1]{\rule{0.5in}{0.2pt} }

\vspace{2ex}
\begin{multicols}{3}
	\begin{enumerate}
		\item $r=4\cos(3\theta)$ \aline{G}
		\item $r=4+2\cos(\theta)$ \aline{E}
		\item $r = 3\csc(\theta)$ \aline{L}
		\item $r=4\sin(3\theta)$ \aline{F}
		\item $r = \theta/2$ \aline{K}
		\item $r=4\sin(\theta)+4\cos(\theta)$ \aline{H}
		\item $r = 1-3\sin(\theta)$ \aline{A}
		\item $r^2=4\sin^2(\theta)$ \aline{I}
		\item $r = 3+2\sin(\theta)$ \aline{C}
		\item $r = 3-2\sin(\theta)$ \aline{D}
		\item $r = 1+3\cos(\theta)$ \aline{B}
		\item $r = 3\sec(\theta)$ \aline{J}
	\end{enumerate}
	\end{multicols}
\clearpage
\section*{Short Answer -- 2 pts each}
{\large Work must be shown for credit.}
\vspace{2ex}
\begin{questions}
	\setlength{\answerlinelength}{2in}
	\setlength{\answerskip}{1.5in}
	\setlength{\answerclearance}{2ex}
	\begin{minipage}{\linewidth}
		\question Convert the polar coordinate to rectangular coordinates: $(-2, 2\pi/3)$
		\answerline[$\left(1,-\sqrt{3}\right)$]

\end{minipage}

\begin{minipage}{\linewidth}
\question Convert the polar coordinate to rectangular coordinates: $(4,-\pi/2)$
\answerline[$\left(0,-4\right)$]
\end{minipage}

\begin{minipage}{\linewidth}



\question Convert the rectangular coordinate to polar coordinates: $(15, 5\sqrt{3})$
\answerline[ $\left(10 \sqrt{3},\frac{\pi }{6}\right)$]
\end{minipage}

\begin{minipage}{\linewidth}



\question Convert the rectangular coordinate to polar coordinates: $(-12, -12)$
 \answerline[$\left(12 \sqrt{2},-\frac{3 \pi }{4}\right)$]

\end{minipage}

\begin{minipage}{\linewidth}



\question Convert the rectangular equation to polar: $x^2 + y^2 = 16$
 \answerline[$r = 4$]
\end{minipage}

\begin{minipage}{\linewidth}



\question Convert the rectangular equation to polar: $2xy=1$
 \answerline[$r^2 = \dfrac{1}{2\sin\theta\cos\theta}$]
\end{minipage}

\begin{minipage}{\linewidth}


\question Convert the polar equation to rectangular: $\theta = 2\pi/3$
 \answerline[$y = -\sqrt3 x$]


\end{minipage}

\begin{minipage}{\linewidth}


\question Convert the polar equation to rectangular: $r = \dfrac{2}{1 + \sin \theta}$
 \answerline[$x^2 + y^2 = (y-2)^2$]
\end{minipage}

\begin{minipage}{\linewidth}

%\question Find the maximum value of $|r|$ for the graph $r = 6 + 12 \cos(\theta - \pi/3)$ and the $\theta$ where it occurs.
%\question Find the maximum value of $|r|$ for the graph $r = \dfrac{1}{3 + 2\sec{\theta}}$ and the $\theta$ where it occurs.
%\question Which of the following are $x$-intercepts of $r = 3(1-2\cos\theta)$?
%\question Which of the following of $y$-intercepts of $r = 4 \sin 3\theta$?

\question Find the intersection points of $r = 3 \cos \theta$ and $r = \sqrt3 \sin \theta$


\answerline[ $\{ \pi/3, 4\pi/3 \}$]
\end{minipage}

\end{questions}

\clearpage
\section*{Free Response Section}
\noindent
\textbf{Calculator Active}
\vspace{2ex}
%{2. Calculator Active - \#1 on 2011 BC Exam}

At time $t$, a particle moving in the $x y$-plane is at
position $(x(t), y(t))$, where $x(t)$ and $y(t)$ are
 not explicitly given. For
 $t \geq 0, \dfrac{d x}{d t}=4 t+1$ and
 $\dfrac{d y}{d t}=\sin \left(t^{2}\right)$.
  At time $t=0,\; x(0)=0$ and $y(0)=-4$.

\begin{enumerate}
\item Find the speed of the particle at time
$t=3$\\[1in]

\item Find the acceleration vector of the particle at time $t=3$.\\[1in]

\item Find the slope of the line tangent to
 the path of the particle at time $t=3$.\\[1in]

\item Find the position of the particle at
time $t=3$.\\[1.5in]

\item Find the total distance traveled by the
particle over the time interval $0 \leq t \leq 3$.
\end{enumerate}
\end{document}